% \begin{algorithmic}
%     \For{i=1,2,3,\dots,n}
%         \State$b_{ii} = a_{ii}$
%
%         \State $b_{ik} = a_{ik}\cos{(\theta)} - a_{il}\sin{(\theta)}$
%
%         \State $b_{il} = a_{il}\cos{(\theta)} + a_{ik}\sin{(\theta)}$
%
%         \State $b_{kk} = a_{kk}\cos^2{(\theta)} -
%         2a_{kl}\cos{(\theta)}\sin{(\theta)} +
%         a_{ll}\sin^2{(\theta)}$
%
%         \State $b_{ll} = a_{ll}\cos^2{(\theta)} -
%         2a_{kl}\cos{(\theta)}\sin{(\theta)} +
%         a_{kk}\sin^2{(\theta)}$
%
%         \State $b_{kl} =
%         (a_{kk}-a_{ll})\cos{(\theta)}\sin{(\theta)} +
%         a_{kl}(\cos^2{(\theta)} - \sin^2{(\theta)})$
%     \EndFor
% \end{algorithmic}

% m_\odot \mathbf{v}_\odot
% + m_\textrm{Mercury} \mathbf{v}_\textrm{Mercury}
% + m_\textrm{Venus} \mathbf{v}_\textrm{Venus}
% + m_\textrm{Earth} \mathbf{v}_\textrm{Earth}
% + \dots + m_= 0

It is desirable that the center of mass of the solar system is in the origin of the coordinate system, i.e. the total momentum $\mathbf{p}$ of all the celestial bodies should be $0$, like in equation \ref{eq:momentum_sum}. Gravity is a conservative force, so if this is true at the initialization it should remain true always. If the initial velocites of all planets (and moons, asteroids, etc., if any) are given, one need simply to solve for the Sun's velocity $\mathbf{v}_\odot$ in equation \ref{eq:momentum_sun} and set the velocity to this number in the program to make sure that $\mathbf{p} = 0$.
\begin{align}
	\label{eq:momentum_sum}
	\mathbf{p} =
	\mathbf{p}_\odot
	+ \mathbf{p}_\textrm{Mercury}
	+ \mathbf{p}_\textrm{Venus}
	+ \mathbf{p}_\textrm{Earth}
	+ \dots
	+ \mathbf{p}_\textrm{Pluto}
	&= 0 \\
	\label{eq:momentum_sun}
	m_\odot \mathbf{v}_\odot
	+ m_1 \mathbf{v}_1
	+ m_2 \mathbf{v}_2
	+ m_3 \mathbf{v}_3
	+ \dots
	+ m_N \mathbf{v}_N
	&= 0
\end{align}
$$
	\mathbf{v}_\odot = - \frac{1}{m_\odot} \sum_i^N m_i \mathbf{v}_i
$$
