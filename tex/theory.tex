One can with good conscience consider gravitation to be the main
governing force in the solar system. Combining this with Newton's
second law of motion gives one a set of equations that can be used
in order to simulate how the solar system behaves in time.

Newton's law of gravity \[ F_G = G\frac{M_1 M_2}{r^2}, \]
formulated in 1687 can be used for calculating the forces acting on
the objects. From there, Newton's second law states that the sum of
the forces acting on an object is proportional to the acceleration
via \[ \Sigma \mathbf{F} = m \mathbf{a}. \] This leads to the
differential equation \[ \frac{\D^2 \mathbf{x}}{\D t^2} =
\mathbf{a} = \frac{\Sigma \mathbf{F}}{M_{\text{obj}}}, \] where
$M_{\text{obj}}$ is the mass of that particular object. The aim is
to solve this equation using numerical methods.

When working with problems on the astronomical scale values tend to
be high, both in masses, velocities and distances. In order to
avoid numerical errors in representation of numbers a better choice
of units than the standard SI units is preferable.

The astronomical unit, labeled AU, where 1 AU is the mean distance
from the earth to the sun, can be used for this particular purpose.
Naturally, velocities can be expressed with the AU / year unit.
Converting from m / s can be done via the formula \[1 \text{ m/s} =
\frac{1}{(1 \text{ AU})_{\text{m}} \cdot (1 \text{ year})_{\text{s}}}. \]

Furthermore, a choice of mass units that lets one ommit the
multiplication with the gravitational constant $G$ would be
preferable. As a result of the near circular orbit of Earth around
the sun, the force between these two objects must obey the relation
\[ F_G = \frac{M_{\text{Earth}}v^2}{r} = \frac{G M_{\odot}
M_{\text{Earth}}}{r^2}. \] Which means that \[G M_{\odot} = 4\pi^2
\text{ AU}^3 / \text{year}^2. \]

In other words, calculating all the masses relative to the sun's
mass means that the gravitational forces easier can be calculated.
The masses must be scaled according to the relation \[ G
    M_{\text{obj}} \text{ AU}^3 / \text{year}^2 = 4\pi^2
    \frac{(M_{\text{obj}})_{\text{kg}}}{(M_{\odot})_{\text{kg}}} \text{ AU}^3
/ \text{year}^2. \]

% Boundary conditions
