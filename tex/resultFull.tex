\bilde{../data/Full_Solar_System__very_long/overview.png}{fig:overview_very_long}{Overview of the solar system.}

In figure \refig{overview_very_long} we can see the result of a simulation where all planets have completed several orbits. The orbits appear to be ellipses with different eccentrisities, which are quite near $0$, like in the real solar system. Pluto has a stronger eccentrisity, which places its orbit closer to the Sun than Neptune's in a small area around its perihelion. Also, PLuto's orbit isn't as static as the proper planets', probably due its low mass and strong eccentrisity, which makes the others planets ``toss it around''.

\bilde{../data/Full_Solar_System__short/inner.png}{fig:inner}{The inner solar system. The outer planets are out of view.}

In figure \refig{inner} we can see the sun with the four inner planets. Note that that especially Earth's orbit is almost perfectly circular (use the grid).

\bildeto{../data/Full_Solar_System__short/sun.png}{../data/Full_Solar_System__long/sun.png}{fig:sun}{The Sun's orbit for different spans of time.}

So far we have primarely studied the orbit of the planets. The Sun also have an orbit, it is just small because of the Sun's enormous mass. Jupiter should be the dominating factor in this small orbit, since Jupiter contains most of the combined mass of the planets. A Jupiter year is $\approx 12$ years. In figure \refig{sun}(top) we see that the Sun completes $\approx 2$ orbits in 25 years, so this seems correct. However the two orbits have very different sizes. This is because of the influence of the other planets. In \refig{sun}(bottom) is the Suns path over a longer period of time. It seems very chaotic, but it is only as chaotic as the combined orbits of the planets. It looks like this because each planet have a different period and gravitational field. If a fourier analysis were performed on this path we would could maybe have extracted the frequencies of some or all of the planets.

\bildeto{figures/earth_path_alone.eps}{figures/earth_path.eps}{fig:earth_path}{Close-up on Earth's orbit with and without other planets present.}
The planets do also behave in this manner,  but it is not so noticeable because they, unlike the Sun, have a clear primary orbit (which is caused by the Sun). In figure \refig{earth_path} a close-up on the Earth's path is shown with and without the presence of other planets. The difference between the extremities of its orbits is $\approx 0.02$, which is also the difference between the Sun's extreme positions, according to figure \refig{sun}(bottom).

The Suns radius is $\approx 0.005 \U{AU}$, which is also roughly the average radius of its orbit according to our results, so a lot of the time the center of mass of the solar system is inside or close to the Sun's surface.
