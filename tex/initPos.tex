In the figures \refig{sunEarthJupiter}-\refig{jupiter1000} the entire solar system is systematically shifting upwards in the $y$-direction. So far, the Sun has been placed with $\V_0 = 0$ in origo and Jupiter (and the Earth) have been given initial velocities in the positive $y$-direction. This has led to the center of mass having a total momentum $\mathbf{p}_y \neq 0$. It is desirable that the total momentum $\mathbf{p}$ of all the celestial bodies should be $0$, like in equation \refeq{momentum_sum}. Then, the solar system will remain in place and all (stable) orbits will remain where they are. This makes it much easier to see what is acually going on when plotting and analyzing the result. In reality, the solar system \emph{do} have a total momentum, but in our model there is nothing moving relative to the solar system, so this movement is mostly troublesome.

Gravity is a conservative force, so if $\mathbf{p} = 0$ is true at the initialization it should remain true always. If the initial velocites of all planets (and moons, asteroids, etc., if any) are given, one need simply to solve for the Sun's velocity $\mathbf{v}_\odot$ in equation \refeq{momentum_sun}, so that we get \refeq{momentum_sum2}, and set the velocity to this number in the program to make sure that $\mathbf{p} = 0$.
\begin{align}
	\label{eq:momentum_sum}
	\mathbf{p} =
	\mathbf{p}_\odot
	+ \mathbf{p}_\textrm{Mercury}
	+ \mathbf{p}_\textrm{Venus}
	+ \mathbf{p}_\textrm{Earth}
	+ \dots
	+ \mathbf{p}_\textrm{Pluto}
	&= 0 \\
	\label{eq:momentum_sun}
	m_\odot \mathbf{v}_\odot
	+ m_1 \mathbf{v}_1
	+ m_2 \mathbf{v}_2
	+ m_3 \mathbf{v}_3
	+ \dots
	+ m_N \mathbf{v}_N
	&= 0
\end{align}
\begin{equation}
    \label{eq:momentum_sum2}
    \mathbf{v}_\odot = - \frac{1}{m_\odot} \sum_i^N m_i \mathbf{v}_i
\end{equation}

The total center of mass (CM) is always close to the Sun, but when $\\mathbf{p} = 0$ the Sun will still move, whilst CM will not, so it is practical to define CM as origo. To do this we must first add the Sun and all planets with with positions that are correct relative to eachother, then calculate which position is the CM and subtract this position from every object, i.e. shifting the entire system so that the CM is placed in origo. This can be done with the following algorithm:
\begin{algorithmic}
    \State $ \mathbf{r}_{\textrm{CM}} = \sum_i^N m_i \mathbf{r}_i $
    \For{i = 1:N}
        \State $ \mathbf{r}_i = \mathbf{r}_i - \mathbf{r}_{\textrm{CM}} $
    \EndFor
\end{algorithmic}

The data on the planets' initial positions and velocities were from NASA. It was difficult to find information on where each planet is at a given time, so we have assumed that when $t=0$ every planet is on its perihelion (and thus have it's greatest orbital velocity) and have given the perihelions arbitrary angular positions. This leads to each planet's orbit being correct relative to the Sun, but not correct relative to the other planets (this is only a problem if our goal is to find the \emph{actual} state of the solar system at various times). The general picture, however, is correct and we can recognize some qualitative details of the actual solar system.
