The aim of the simulation is to solve equation
\refeq{newtonDifferential}. This will be done by employing the
Runge-Kutta 4th order numerical integration scheme.

% Choice of numerical method
When solving this kind of differential problems, the choice of
integration scheme is of severe importance. Choosing a less
accurate method might provide incorrect solutions and if the
problem has exceedingly rapid changes a more stable (implicit)
integration scheme must be considered.

Considering the huge time scale, even with the scaled units, this
problem will not provide solutions with such rapid changes.
Therefore it is sufficient to choose the explicit RK4 scheme. The
local truncation error is on the order of $O(\D t^5)$ and the global
error is on the order of $O(\D t^4)$. Again, considering that the
time is of unit years, this error is small ``enough'' as to avoid
monumental errors in the solutions.

This means that for example choosing $\D t = 1 \cdot 10^{-2}$ years
will produce a global error proportional to $\D t^4 = 1 \cdot
10^{-8}$ years which is approximately 0.32 seconds.
% End choice of numerical method

% How does the method work
RK4 is a widely used integration scheme. It determines the next
step in the integration from the previous value and a weighted
average based on four different approximations found by applying
Euler's method and the midpoint method.

In other words, provided you have the solution $\R_n$ at the
$n$-th point. The next point $\R_{n+1}$ is found via the expression
\[ \R_{n+1} = \R_n + \frac{1}{6}(k_1 + 2k_2 + 2k_3 + k_4), \] where 
$k_1,k_2,k_3,k_4$ are the following approximations,
\begin{align*}
    k_1 &= f(\R_n), \\
    k_2 &= f(\R_n + \frac{1}{2}\D t k_1), \\
    k_3 &= f(\R_n + \frac{1}{2}\D t k_2), \\
    k_4 &= f(\R_n + \D t k_3),
\end{align*}
and $f(\R)$ is the right hand side of the differential equation.
From these expressions one can see that $k_1$ represent a regular
Forward Euler step, $k_2$ is the increment at half the Forward
Euler step (a midpoint method step), $k_3$ is also a midpoint
method step but now from $k_2$. And lastly $k_4$ is a Forward Euler
step from $k_3$. A weighted average of these provide a reasonable
approximation to the next step.
% Why does the combination of these steps provide a good
% approximation?
%
% End of how the method works

% On the rewriting into set of first order equations
The RK4 scheme represents a first order integration whereas the
problem to be solved is a second order differential equation.
Therefore, in order to use this scheme, the problem must be
rewritten into a set of first order differential equations. These
can then be solved in parallel throughout an iteration.

Using the fact that the acceleration can be calculated via the
forces acting on the object the problem can be rewritten from
\refeq{newtonDifferential} into
\begin{align}
    \frac{\D \mathbf{r}}{\D t} &= \mathbf{v}, \label{diffSetFirstR} \\
    \frac{\D \mathbf{v}}{\D t} &= \mathbf{a} = \frac{\Sigma
        \mathbf{F}}{M}.
    \label{diffSetFirstV}
\end{align}
% End of rewriting into set of first order equations

% On how this can be represented object oriented
When implementing this problem, dividing into objects and solving
via object oriented programming is a wise approach. In this way, if
the need for a simulation of many objects arises, no (or only very
small) changes have to be done in the source code itself.

This was done by creating classes to represent
\begin{itemize}
    \item A celestial object
    \item A whole system
\end{itemize}
The instances of a celestial object would then contain describing
attributes like mass, position and velocity while the system
contained a list of the celestial objects and methods for advancing
the system in time, finding forces between the objects and so on.
% End of how this can be object oriented

% Overview algorithm
This leads to the following algorithm for advancing the solar system
in time.
\begin{algorithmic}
    \For{t=0,dt,\dots,T}
        \For{i=Sun,Earth,\dots,Pluto}
            \State $(K1)_{\V} = \frac{\Sigma \mathbf{F}_i}{M_i}$
            \State $(K1)_{\R} = \R_i$
        \EndFor

        \State $\V_{\text{all}} = \V + \frac{1}{2}(K1)_{\V} \D t$
        \State $\R_{\text{all}} = \R + \frac{1}{2}(K1)_{\R} \D t$

        \For{i=Sun,Earth,\dots,Pluto}
            \State $(K2)_{\V} = \frac{\Sigma \mathbf{F}_i}{M_i}$
            \State $(K2)_{\R} = \R_i$
        \EndFor

        \State $\V_{\text{all}} = \V + \frac{1}{2}(K2)_{\V} \D t$
        \State $\R_{\text{all}} = \R + \frac{1}{2}(K2)_{\R} \D t$

        \For{i=Sun,Earth,\dots,Pluto}
            \State $(K3)_{\V} = \frac{\Sigma \mathbf{F}_i}{M_i}$
            \State $(K3)_{\R} = \R_i$
        \EndFor

        \State $\V_{\text{all}} = \V + (K3)_{\V} \D t$
        \State $\R_{\text{all}} = \R + (K3)_{\R} \D t$

        \For{i=Sun,Earth,\dots,Pluto}
            \State $(K4)_{\V} = \frac{\Sigma \mathbf{F}_i}{M_i}$
            \State $(K4)_{\R} = \R_i$
        \EndFor

        \State $\V_{\text{all}} = \V + \frac{1}{6}\left[ (K1)_{\V}
            + 2(K2)_{\V} + 2(K3)_{\V} + (K4)_{\V} \right] \D t$
        \State $\R_{\text{all}} = \R + \frac{1}{6}\left[ (K1)_{\R}
            + 2(K2)_{\R} + 2(K3)_{\R} + (K4)_{\R} \right] \D t$
    \EndFor
\end{algorithmic}
Since each object depends on a force from all other objects, one
cannot simply perform the integration for one object at a time ---
when calculating the forces to find, for example $k_2$, the whole
system has to be moved to the instance where all the objects'
positions are moved via a Forward Euler step. This means that
several loops through all the objects is necessary. Fortunately,
this can be done in few lines of code provided an object oriented
approach is chosen.
% End overview algorithm
